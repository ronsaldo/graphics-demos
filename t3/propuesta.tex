\documentclass{article}

\usepackage[utf8]{inputenc}
\usepackage[spanish]{babel}
\usepackage{amsmath}
\usepackage{amsfonts}
\usepackage{amssymb}

\begin{document}
\title{Visualización de Terreno Mediante Ray-Tracing en GPU}
\author{Ronie Salgado}
\date{}

\maketitle

El objetivo de este proyecto consiste en crear un ray-tracer sencillo en GPU
para explorar el uso de la API de OpenCL. Para ello, se pretende visualizar
un terreno sencillo representado mediante un {\it height-map}, el cual puede
ser generado de diversos modos. Se explorara en un primer objetivo la generación
del terreno a partir de herramientas como {\it Perlin Noise}\cite{perlin}, un generador
pseudo aleatorio, diferenciable y cuyos valores son ``deterministas'' para una 
posición dada en el espacio. \\

Como punto de partida se utilizara un tutorial disponible en el archivo del
sitio web {\it flipcode}\cite{raytrace}. Adaptando adecuadamente el algoritmo para
usarlo en GPU con un solo objeto a visualizar, un terreno representado que se
representa como una grilla rectangular de la que solo se almacenan las alturas.
Para ello, se pretende utilizar una técnica similar a la usada en
{\it Parallax-Mapping}. \\

En adición a ello, se pretende experimentar un poco con la generación de cielo,
aplicando técnicas existentes para ello\cite{gpugems}. \\

La plataforma a utilizar es el lenguaje de programación C++, sobre Linux Mint
64-bit, SDL 1.2 para la creación de una ventana, OpenCL para programar y
comunicarse con la GPU y CMake para la construcción del proyecto. \\

No se pretende buscar un rendimiento elevado, ni muchos cuadros por segundo,
solo experimentar con una demo. La ventaja de usar {\it raytracing} se encuentra
en la trivialidad de implementar sombras realistas, reflexión y refracción con
esa técnica.


\begin{thebibliography}{9}

\bibitem{perlin} Kern Perlin, \emph{Making Noise} Accedido en 2013, disponible en:
         {\it http://www.noisemachine.com/talk1/index.html}
\bibitem{gpugems}
  Sean O'Neil Leslie Lamport, \emph{GPU Gems 2, Chapter 16.} NVIDIA Corporation, 2004.
        Disponible en: {\it http://http.developer.nvidia.com/GPUGems2/gpugems2\_chapter16.html}
\bibitem{raytrace}
 Jacco Bikker , \emph{Raytracing Topics \& Techniques}. 2004. Disponible en: {\it http://www.flipcode.com/archives/Raytracing\_Topics\_Techniques-Part\_1\_Introduction.shtml}
\end{thebibliography}

\end{document}

